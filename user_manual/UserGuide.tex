% Manual de usuario
% Modificado por Crisostomo Barajas a partir del manual creado por Emmanuel Martinez
% Julio 2022

\documentclass[12pt,oneside,a4]{ol-softwaremanual}
\usepackage{float}
\usepackage[utf8]{inputenc}
\usepackage[a4paper,width=160mm,top=25mm,bottom=25mm]{geometry}
\usepackage[lining,tabular]{fbb} % so math uses tabular lining figures
\usepackage{graphicx}
\usepackage{enumitem}
\setlist{leftmargin=*}
\usepackage{listings}
\lstset{basicstyle=\ttfamily,frame=single,xleftmargin=3em,xrightmargin=3em}
\usepackage[os=win]{menukeys}
\renewmenumacro{\keys}[+]{shadowedroundedkeys}
\usepackage{framed}
\usepackage{etoolbox}
\AtBeginEnvironment{leftbar}{\sffamily\small}
\usepackage{array,lipsum}
\newenvironment{fulltable}[1][H]
 {\begin{table}[#1]%
  \hspace*{-\leftmarginwidth}%
  \begin{minipage}{\fullwidth}}
 {\end{minipage}\end{table}}
\usetikzlibrary{chains,arrows,shapes,positioning}
\usepackage{hyperref}
\graphicspath{{figures/}} %Setting the graphicspath
\renewcommand\abstractname{Introduction}

\usepackage[spanish]{babel}

\usepackage{multicol}

\usepackage{caption}
\usepackage[rightcaption]{sidecap}
\usepackage{subcaption}
\usepackage{wrapfig}

\usepackage{fontawesome}

\usepackage{amsmath}
\usepackage{amssymb}

\definecolor{ClearPurple}{RGB}{136, 0, 255}

\usepackage{tikz}
\newcommand*\circled[1]{\tikz[baseline=(char.base)]{
            \node[shape=circle,draw,inner sep=2pt] (char) {#1};}}

\newenvironment{Figure}
  {\par\medskip\noindent\minipage{\linewidth}}
  {\endminipage\par\medskip}

\title{\vspace{20mm} \large{Manual de Usuario}\\ \vspace{10mm} \huge{Diseño de Adquisición Sísmica Compresiva}\\ \vspace{5mm} \huge{ReDs}}
\author{Emmanuel Martínez \\ Crisóstomo A. Barajas-Solano}
\softwarelogo{\includegraphics[width=10cm]{figures/applogo}}
\version{2022.7}



\begin{document}

\maketitle
\begin{center}
\includegraphics[width=.9\linewidth]{header.png}
\end{center}
\begin{abstract}
%\includegraphics[width=1.0\linewidth]{PhysLogger.png}\\
%Esta aplicación pertenece al proyecto de investigación 9836, de la Universidad Industrial de Santander, Colombia. Esta aplicación permite la reconstrucción de trazas de muestras sísmicas mediante 4 diferentes algoritmos. Esta reconstrucción puede ser configurada por el usuario de tal manera que pueda realizar diferentes tipos de submuestreo y ajuste de parámetros mediante una interfaz gráfica de usuario.
Esta herramienta software hace parte del proyecto 9836 - ``Nuevas tecnologías computacionales para el diseño de sistemas de adquisición sísmica 3D terrestre con muestreo compresivo para la reducción de costos económicos e impactos ambientales en la exploración de hidrocarburos en cuencas terrestres colombianas'' adscrito a la Convocatoria para la financiación de proyectos de investigación en geociencias para el sector de hidrocarburos, desarrollado por la alianza, Universidad Industrial de Santander (UIS), ECOPETROL y la Asociación Colombiana de Geólogos y Geofísicos del Petróleo (ACGGP). Este proyecto es financiado por MINCIENCIAS y la Agencia Nacional de Hidrocarburos (ANH). Los derechos sobre este software están reservados a las entidades aportantes.
\end{abstract}

\clearpage
\tableofcontents

\clearpage

\section{Visión General del Software}

Esta aplicación permite la reconstrucción de trazas de muestras sísmicas mediante 4 diferentes algoritmos pueden ser configurados por el usuario de tal manera que pueda realizar diferentes tipos de submuestreo y ajuste de parámetros mediante una interfaz gráfica de usuario, esto se observa en la figura \ref{fig:vision}.

La aplicación se encuentra distribuida en 7 partes: \circled{1} la barra de menú, \circled{2} Lector de datos, \circled{3} panel de algoritmos, \circled{4} panel de submuestreo, \circled{5} menú de experimentos \circled{6} menú de cambio de vista y finalmente, \circled{7} menú de visualización.

\begin{Figure}
    \centering
    \includegraphics[width=1\linewidth]{main.png}
     \captionof{figure}{Partes de la interfaz general de la aplicación.}
    \label{fig:vision}
\end{Figure}

A continuación se mostrará de forma especifica las herramientas principales de la aplicación y, las diferentes configuraciones y opciones que pueden ser aplicadas a las muestras sísmicas para la realización de diferentes tipos de experimentos. Cabe resaltar que de acuerdo al tipo de prueba que se quiera realizar \circled{1}, el comportamiento de algunos de los paneles disponibles cambiará para adecuarse a dicha tarea.


\begin{multicols}{2}
\end{multicols}


\section{Menú Principal}

La aplicación cuenta con una barra de tareas, que se encuentra en la parte superior \circled{1} de la figura \ref{fig:vision}. Aquí se observan 3 botones distintos que permitirán realizar distintos tipos de pruebas sobre los datos sísmicos.


\begin{multicols}{2}

Como se observa en la figura \ref{fig:main_button}, el primer botón habilitará a la aplicación para realizar pruebas generales sobre las muestras sísmicas, este es el menú principal.

\begin{Figure}
    \centering
    \includegraphics[width=0.3\linewidth]{main-tab.png}
     \captionof{figure}{Botón del menú principal.}
    \label{fig:main_button}
\end{Figure}

\end{multicols}

El menú principal permite realizar experimentos con las muestras sísmicas siguiendo el orden que se observa en la figura \ref{fig:vision}. \\

Ahora se observará paso a paso como funcionan cada uno de estos paneles, desde la lectura de datos sísmicos hasta la visualización y guardado de resultados.


\subsection{Lectura de Datos}

Como se observa a la izquierda \circled{2} de la figura \ref{fig:vision}, el panel de lectura de datos mostrará los nombres de los datos sísmicos que el usuario haya cargado a la aplicación para realizar diversas pruebas.\\

\textbf{Leyendo una muestra sísmica} \label{sec:data_lecture}

\begin{multicols}{2}
Para leer una muestra sísmica, se pulsa la opción \emph{Cargar}, como se observa en la figura \ref{fig:data_lecture_1}. Inmediatamente se abrirá la ventana que se observa en la figura \ref{fig:data_lecture_2}, donde el usuario podrá seleccionar un dato sísmico (en este caso \emph{data.npy}, también se pueden cargar datos \emph{.mat}) que es el tipo actual de dato que lee la aplicación.

\begin{Figure}
    \centering
    \includegraphics[width=.7\linewidth]{data-lecture-1.png}
     \captionof{figure}{Cargando una muestra sísmica.}
    \label{fig:data_lecture_1}
\end{Figure}

\end{multicols}

\begin{Figure}
    \centering
    \includegraphics[width=1\linewidth]{data-lecture-2.png}
     \captionof{figure}{Ventana de selección de dato sísmico.}
    \label{fig:data_lecture_2}
\end{Figure}

\begin{multicols}{2}

Una vez seleccionado, se debe pulsar en la opción \emph{Abrir} y el dato sísmico cargado se podrá observar en la aplicación como se muestra en la figura \ref{fig:data_lecture_3}. Aquí los datos se observan en el siguiente orden, \circled{I} reprenta al directorio padre y \circled{II} es el dato sísmico cargado hijo del directorio padre.

\begin{Figure}
    \centering
    \includegraphics[width=0.7\linewidth]{data-lecture-3.png}
     \captionof{figure}{Panel con un dato sísmico cargado.}
    \label{fig:data_lecture_3}
\end{Figure}

\end{multicols}

\subsection{Algoritmos}

Los algoritmos disponibles en la aplicación son \emph{FISTA}, \emph{GAP}, \emph{TwIST} y \emph{ADMM}, como se observa en la parte izquierda \circled{3} de la figura \ref{fig:vision}. En este panel se pueden observar diversas opciones que permitirán configurar los parámetros de acuerdo al algoritmo seleccionado, incluyendo la cantidad máxima de iteraciones a realizar. Adicionalmente, el botón \hspace{0.5mm} \faEye \hspace{0.5mm} permite visualizar los diferentes algoritmos con sus correspondientes parámetros, como se observa en la figura \ref{fig:algorithms}.

\begin{figure}[!ht]
     \centering
     \begin{subfigure}[b]{0.45\textwidth}
         \centering
         \includegraphics[width=\textwidth]{algorithm-fista.png}
         \caption{FISTA}
         \label{fig:fista}
     \end{subfigure}
     \hfill
     \begin{subfigure}[b]{0.45\textwidth}
         \centering
         \includegraphics[width=\textwidth]{algorithm-admm.png}
         \caption{ADMM}
         \label{fig:admm}
     \end{subfigure}
     \begin{subfigure}[b]{0.45\textwidth}
         \centering
         \includegraphics[width=\textwidth]{algorithm-gap.png}
         \caption{GAP}
         \label{fig:gap}
     \end{subfigure}
     \hfill
     \begin{subfigure}[b]{0.45\textwidth}
         \centering
         \includegraphics[width=\textwidth]{algorithm-twist.png}
         \caption{TwIST}
         \label{fig:twist}
     \end{subfigure}
        \caption{Algoritmos disponibles y sus respectivos parámetros.}
        \label{fig:algorithms}
\end{figure}

Cada algoritmos tendrá una configuración de parámetros por defecto que el usuario podrá configurar con desee.

\subsection{Submuestreo}\label{subsampling}

Debido a que hay muchas formas de recuperar un \emph{shot} a partir de las medidas recolectadas previamente, en este panel se podrá configurar que tipo de submuestreo se desea realizar a la muestra sísmica a manera de simulación, como se observa en la parte izquierda \circled{4} de la figura \ref{fig:vision}. Todos los tipos de submuestreo excepto el de tipo lista comparten el mismo nivel de compresión que el usuario desee asignar. A continuación se observarán que tipos de submuestreo se pueden aplicar:\\

\underline{\textbf{Aleatorio}}

\begin{multicols}{2}

Aplica un submuestreo aleatorio a las muestras sismicas de acuerdo al nivel de compresión. Al habilitar la semilla, se puede establecer el mismo submuestreo para todas las veces que se ejecute un experimento, como se observa en la figura \ref{fig:subsampling_random}.

\begin{Figure}
    \centering
    \includegraphics[width=0.8\linewidth]{subsampling-random.png}
     \captionof{figure}{Submuestreo aleatorio.}
    \label{fig:subsampling_random}
\end{Figure}

\end{multicols}

\underline{\textbf{Uniforme}}

\begin{multicols}{2}

Aplica un submuestreo uniforme a las muestras sismicas de acuerdo al nivel de compresión, como se observa en la figura \ref{fig:subsampling_uniform}.

\begin{Figure}
    \centering
    \includegraphics[width=0.8\linewidth]{subsampling-uniform.png}
     \captionof{figure}{Submuestreo uniforme.}
    \label{fig:subsampling_uniform}
\end{Figure}

\end{multicols}

\underline{\textbf{Jitter}}

\begin{multicols}{2}

Aplica un submuestreo aleatorio o uniforme de acuerdo a la cantidad de bloques ingresados por el usuario a las muestras sismicas de acuerdo al nivel de compresión, como se observa en la figura \ref{fig:subsampling_jitter}.

\begin{Figure}
    \centering
    \includegraphics[width=0.8\linewidth]{subsampling-jitter.png}
     \captionof{figure}{Submuestreo jitter.}
    \label{fig:subsampling_jitter}
\end{Figure}

\end{multicols}

\underline{\textbf{Lista}}

\begin{multicols}{2}

Aplica un submuestreo de tipo lista donde los \emph{shots} a remover son ingresados por el usuario a las muestras sismicas, como se observa en la figura \ref{fig:subsampling_list}. Para este tipo de submuestreo es necesario validar que el usuario ingrese los datos con un formato lógico y especifico.

\begin{Figure}
    \vspace{5mm}
    \centering
    \includegraphics[width=0.8\linewidth]{subsampling-list.png}
     \captionof{figure}{Submuestreo de tipo lista.}
    \label{fig:subsampling_list}
\end{Figure}

\end{multicols}

Las condiciones son las siguientes:

\begin{multicols}{2}

\begin{itemize}
    \item La secuencia deben ser $x_1,x_2,x_n$, tal que $x_i \in \mathbb{Z}^+$, donde entre cada $x_i$ no deben haber espacios y estar separados por números, como se observa en la figura \ref{fig:subsampling_list_1}.
    \item Cada $x_i$ debe cumplir que $0 \leq x_i \leq N$, donde $N$ es la cantidad máxima lineas de receptores que contiene un \emph{shot}.
    \item Deben haber mínimo siete $x_i$ distintos.
\end{itemize}

\begin{Figure}
    \vspace{5mm}
    \centering
    \includegraphics[width=0.8\linewidth]{subsampling-list-1.png}
     \captionof{figure}{Ejemplo de submuestreo de tipo lista.}
    \label{fig:subsampling_list_1}
\end{Figure}

\end{multicols}

\subsection{Experimentos}
\label{sec:experiment}

Una vez ya se haya cargado un dato sísmico, seleccionado el algoritmo, ajustado sus parámetros y configurado el tipo de submuestreo, entonces se podrá realizar un experimento. Como se observa en la parte izquierda \circled{5} de la figura \ref{fig:vision}.

\subsubsection{Realizando un experimento}

\begin{multicols}{2}

Para iniciar un nuevo experimento se debe pulsar el botón \hspace{0.5mm} \faSave \hspace{0.5mm} que se observa en la figura \ref{fig:experiment_1}. Aquí el usuario seleccionará el directorio donde desee que los resultados de su experimento sean guardados, como se observa en la figura \ref{fig:experiment_2}.

\begin{Figure}
    \vspace{5mm}
    \centering
    \includegraphics[width=0.8\linewidth]{experiment-1.png}
     \captionof{figure}{Panel de experimentos.}
    \label{fig:experiment_1}
\end{Figure}

\end{multicols}

\begin{Figure}
    \centering
    \includegraphics[width=1\linewidth]{experiment-2.png}
     \captionof{figure}{Ventana de guardado de resultados.}
    \label{fig:experiment_2}
\end{Figure}

\begin{multicols}{2}

Para poder guardar el experimento es necesario asignarle un nombre, en este ejemplo se pondrá \emph{nuevo\_experimento}, este archivo será guardado con la extensión \emph{.npz} y finalmente, presionamos el botón \emph{Guardar}.

Finalmente, para correr el experimento, se debe pulsar en el botón \hspace{0.5mm} \faPlay \hspace{0.5mm} y en la barra de progreso a la izquierda de dicho botón se podrá ir observando en tiempo real cuanto demorá el experimento en ejecutarse, como se observa en la figura \ref{fig:experiment_4}.

\begin{Figure}
    \centering
    \includegraphics[width=0.7\linewidth]{experiment-4.png}
     \captionof{figure}{Ejecución de un experimento en tiempo real.}
    \label{fig:experiment_4}
\end{Figure}

\end{multicols}

\subsection{Visualización de Resultados}

\begin{multicols}{2}

Finalmente, los resultados de todos los experimentos que se realicen a través de esta aplicación se podrán observar a la derecha \circled{7} de la figura \ref{fig:vision}. En este panel se encuentran dos tipos de visualización: la visualización del rendimiento (iteraciones vs. error/psnr) y visualización de la reconstrucción, como se observa en la figura \ref{fig:main_result_1}.

Como se mencionó en la sección \ref{sec:experiment}, los experimentos que sean ejecutados se podrá ver tanto el rendimiento como la reconstrucción de las trazas en tiempo real. A continuación se detallará que es lo que se observa exactamente en cada una de estos subpaneles.

\begin{Figure}
    \vspace{1cm}
    \centering
    \includegraphics[width=1\linewidth]{main-result-1.png}
     \captionof{figure}{Panel de resultados.}
    \label{fig:main_result_1}
\end{Figure}

\end{multicols}

\subsubsection{Rendimiento}

En esta subventana se observará el rendimiento actual del experimento que esté corriendo, como se observa en la figura \ref{fig:main_result_2}. En esta figura se puede observar una gráfica con dos ejes, el \textcolor{red}{eje izquierdo} representa el indice de similaridad estructural (SSIM) y el \textcolor{blue}{eje derecho} representa la proporción máxima de señal a ruido (PSNR) de las trazas reconstruidas con respecto a las verdaderas. En general, ambas métricas son de percepción que cuantifican la degradación de la calidad de la imagen.

Sobre está gráfica que observan diferentes herramientas que pueden ser usadas para modificar alguna carácteristica de la gráfica \ref{fig:main_result_3}, que se dan en este orden:

\begin{multicols}{2}

\begin{Figure}
    \vspace{1cm}
    \centering
    \includegraphics[width=0.8\linewidth]{main-result-3.png}
     \captionof{figure}{Opciones de la gráfica.}
    \label{fig:main_result_3}
\end{Figure}

\begin{Figure}
    \centering
    \includegraphics[width=0.7\linewidth]{main-result-4.png}
     \captionof{figure}{Configuración de subplots.}
    \label{fig:main_result_4}
\end{Figure}

\end{multicols}


\begin{multicols}{2}

\begin{itemize}
    \item[I.]  El botón \hspace{0.5mm} \faHome \hspace{0.5mm} permite volver a la gráfica a su estad original, y las otros dos botones \hspace{0.5mm} \faArrowLeft \hspace{0.5mm}, \hspace{0.5mm} \faArrowRight \hspace{0.5mm} permiten ir hacia adelante o hacia atrás en las modificaciones que se le hayan hecho a la gráfica.
    
    \item[II.] El botón \hspace{0.5mm} \faArrows \hspace{0.5mm} permite desplazar la gráfica en cualquier dirección, y el botón \hspace{0.5mm} \faSearch \hspace{0.5mm} permite realizar un acercamiento o alejamiento de la gráfica.
    
    \item[III.] Los siguientes botones tienen un comportamiento más complejo. El botón \hspace{0.5mm} \faBars \hspace{0.5mm} abre la subventana que se observa en la figura \ref{fig:main_result_4}. Esto permitirá ajustar la visualización de los bordes de la gráfica y el espaciado con respecto a su contenedor padre. Por otra parte, el botón \hspace{0.5mm} \faLineChart \hspace{0.5mm} permitirá configurar la visualización de los ejes (cantidad de valores) y el tipo de escala para cada eje, como se observa en la figura \ref{fig:main_result_5}.
    
    \item[IV.] Finalmente, el botón \hspace{0.5mm} \faSave \hspace{0.5mm} permitirá guardar una imagen de la gráfica exactamente como se observe en ese instante.
    
\end{itemize}

\begin{Figure}
    \centering
    \includegraphics[width=1\linewidth]{main-result-5.png}
     \captionof{figure}{Editor de ejes, curvas y parámetros de la gráfica.}
    \label{fig:main_result_5}
\end{Figure}

\end{multicols}

\begin{Figure}
    \vspace{1cm}
    \centering
    \includegraphics[width=0.8\linewidth]{main-result-2.png}
     \captionof{figure}{Rendimiento de un experimento realizado.}
    \label{fig:main_result_2}
\end{Figure}

\subsubsection{Reconstrucción de Trazas}

Por otra parte, los resultados cuantitativos del experimento se observan en el subpanel de \emph{Reporte} de la figura \ref{fig:main_result_6}. En esta figura, la gráfica se divide en 4 partes:

\begin{multicols}{2}

\begin{itemize}
    \item[I.]  Esta gráfica es la muestra sísmica completa, la cual se usa como referencia para comparar contra la reconstruida, tanto cuantitativa como cualitativamente.
    
    \item[II.] Está gráfica es la muestra sísmica submuestreada y es la que pasa a través de los algoritmos para reconstruir el sensado de las líneas de receptores faltantes.
    
    \item[III.] Está gráfica contiene la muestra sísmica reconstruida.
    
    \item[IV.] Finalmente, esta gráfica contiene una de las trazas reconstruidas comparada contra la de referencia.
    
\end{itemize}

\end{multicols}

\begin{Figure}
    \centering
    \includegraphics[width=1\linewidth]{main-result-6.png}
     \captionof{figure}{Reporte cuantitativo de un experimento realizado.}
    \label{fig:main_result_6}
\end{Figure}

\section{Ajuste de parámetros}

\begin{multicols}{2}

Como se observa en la figura \ref{fig:tuning_button}, el segundo botón habilitará a la aplicación para realizar pruebas basada en un ajuste de parámetros sobre las muestras sísmicas, este es el menú de ajuste de parámetros.

\begin{Figure}
    \centering
    \includegraphics[width=0.3\linewidth]{main-tab.png}
     \captionof{figure}{Botón del menú de ajuste de parámetros.}
    \label{fig:tuning_button}
\end{Figure}

\end{multicols}

El ajuste de parámetros permite probar diferentes rangos de parámetros para cada uno de los algoritmos de una forma sencilla y visualizar la curva de rendimiento en dichos ajustes.

Como se explicó previamente, la aplicación se adapta a los tipos de pruebas que se quieran realizar, por lo que la aplicación deshabilita los paneles u opciones que no son necesarias para el ajuste de parámetros, en este caso, y se habilita un nuevo panel llamado \emph{Ajuste de parámetros}, como se observa en la figura \ref{fig:tuning}.

\begin{Figure}
    \centering
    \includegraphics[width=1\linewidth]{tuning.png}
     \captionof{figure}{Partes del ajuste de parámetros en la aplicación.}
    \label{fig:tuning}
\end{Figure}

En esta figura se observan los cambios que surgen comparados con la interfaz general \ref{fig:vision}. En \circled{1} se deshabilitan los parámetros de los algoritmos para usar el nuevo de panel de ajuste de parámetros \circled{2} y poder realizar nuevas configuraciones de experimentos sobre los datos sísmicos cargados. Adicionalmente, la visualización de resultados se simplifica a una sola gráfica \circled{3}. A continuación, se analizará el panel de ajuste de parámetros y la visualización de resultados.

\subsection{Configuraciones}

Como se observa en la figura \ref{fig:tuning_panel}, el panel de ajuste de parámetros permite al usuario configurar el tipo de prueba que desea realizar y se encuentra dividido en:

\begin{multicols}{2}

\begin{itemize}
    \item[I.] \textbf{Tipo de ajuste:} El tipo de ajuste permite configurar la cantidad de valores de un mismo parámetro para algún algoritmo con los que se desean realizar pruebas. Este puede ser en intervalo, donde el usuario debe ingresar la cantidad de \textcolor{ClearPurple}{\emph{valores}} que desea ingresar, como se observa en la figura \ref{fig:tuning_panel}. El otro tipo de ajuste es de lista, donde el usuario debe ingresar cada uno de los valores con los que desea realizar pruebas.
    
    \item[II.] \textbf{Párametro:} Esta caja de parámetros se adaptará al algoritmo escogido. Su función es permitir al usuario seleccionar el parámetro al cual se le va a realizar el ajuste de acuerdo al tipo de ajuste.
    
    \item[III.] \textbf{Escala:} Permite al usuario modificar el comportamiento de los valores de los parámetros. Esta escala puede ser lineal, cuyo comportamiento no afectará a los valores de los parámetros, de tal manera que $y = x$, donde $x$ es cualquier valor de los parámetros. El otro tipo de escala es la escala logaritmica, la cual tomará cada uno de los valores establecidos por el usuario, de tal manera que $y = 10^x$.
    
\end{itemize}

\begin{Figure}
    \centering
    \includegraphics[width=0.9\linewidth]{tuning-panel.png}
     \captionof{figure}{Panel de ajuste de parámetros.}
    \label{fig:tuning_panel}
\end{Figure}

\end{multicols}

\subsubsection{Realizando un ajuste de parámetros óptimo}

Para realizar un ajuste de parámetros óptimo, primero se deberá cargar el dato sísmico, el proceso es igual a la mencionada en la sección \ref{sec:data_lecture}.

\begin{multicols}{2}

Ahora se debe seleccionar el algoritmo al cual se le desea hallar los parámetros óptimos dado el dato sísmico cargado, en este caso se selecciona FISTA, y se establece la configuración de parámetros que se observa en la figura \ref{fig:tuning_params}. A continuación, se explicará esta selección de parámetros:

\begin{itemize}
    \item Para el ajuste de parámetros se seleccionó de tipo intervalo con escala logarítmica ya que facilita al usuario la selección de parámetros y visualización de resultados obtenidos.
    \item Para el submuestreo se fijó una semilla con el propósito de que la muestra sísmica siempre tenga el mismo tipo de submuestreo y poder hacer una óptima selección de los parámetros.
\end{itemize}

\begin{Figure}
    \centering
    \includegraphics[width=1\linewidth]{tuning-params.png}
     \captionof{figure}{Ajuste de parámetros para el algoritmo FISTA.}
    \label{fig:tuning_params}
\end{Figure}

\end{multicols}

Primero se ajustará el parámetro $\lambda$, estos resultados se observan en la figura \ref{fig:tuning_1}. En esta figura se pueden observar los resultados obtenidos para la configuración de parámetros establecida en la figura \ref{fig:tuning_params}. Por inspección, el valor óptimo para $\lambda$ es $\lambda^* \approx 2.91$.

\begin{Figure}
    \centering
    \includegraphics[width=1\linewidth]{tuning-1.png}
     \captionof{figure}{Ajuste del parámetro $\lambda$ para el algoritmo FISTA.}
    \label{fig:tuning_1}
\end{Figure}

\begin{multicols}{2}

Ahora, para encontrar el valor óptimo $\mu^*$, se toma el valor de $\lambda^*$ y se evalua para un intervalo de $\mu$, como el que se observa en la figura \ref{fig:tuning_params_2}. Los resultados obtenidos para este ajuste de parámetros se observan en la figura \ref{fig:tuning_2}. Por inspección, se observa que el valor óptimo $\mu^* \approx 0.39$.

\begin{Figure}
    \centering
    \includegraphics[width=0.8\linewidth]{tuning-params-2.png}
     \captionof{figure}{Ajuste del parámetro $\mu$ para el algoritmo FISTA.}
    \label{fig:tuning_params_2}
\end{Figure}

\end{multicols}

\begin{Figure}
    \centering
    \includegraphics[width=1\linewidth]{tuning-2.png}
     \captionof{figure}{Ajuste del parámetro $\mu$ para el algoritmo FISTA.}
    \label{fig:tuning_2}
\end{Figure}

\section{Visualización general de Resultados}

Los experimentos realizados quedan almacenados en el directorio escogidos por el usuario de forma local en su computadora. Esos resultados pueden ser visualizados a través del panel de visualización \circled{6} en la figura \ref{fig:vision}.

\subsection{Menú principal}

\begin{multicols}{2}

Esta herramienta funciona de la misma manera para los dos tipos de experimentos actuales (general y ajuste de parámetros), cargando los respectivos datos como se visualizaron en secciones anteriores. Para usar este panel debemos pulsar el botón en \circled{1}, como se observa en la figura \ref{fig:report_1}. Al pulsar este botón el comportamiento de la interfaz cambiará a lectura de resultados de experimentos ya realizados. Por lo que ahora se usará el botón \emph{Cargar} \circled{2} para cargar dichos datos.

\begin{Figure}
    \centering
    \includegraphics[width=0.8\linewidth]{report-1.png}
     \captionof{figure}{Lectura de resultados.}
    \label{fig:report_1}
\end{Figure}

\end{multicols}

\textbf{Leyendo resultados de experimentos realizados}

\begin{multicols}{2}

Como se observa en la figura \ref{fig:report_2}, el usuario podrá seleccionar el archivo que desee visualizar (en ester caso \emph{resultados\_1.npz}) con los resultados de un experimento previamente realizado. Una vez seleccionado, se debe pulsar la opción \emph{Abrir} y los resultados serán cargados tanto en el panel de resultados de la figura \ref{fig:report_3}, donde \circled{1} ese el directorio padre del los resultados cargados \circled{2}.

\begin{Figure}
    \centering
    \includegraphics[width=0.8\linewidth]{report-3.png}
     \captionof{figure}{Panel con resultados cargados de un experimento previo.}
    \label{fig:report_3}
\end{Figure}

\end{multicols}

\begin{Figure}
    \centering
    \includegraphics[width=0.9\linewidth]{report-2.png}
     \captionof{figure}{Ventana de selección de resultados.}
    \label{fig:report_2}
\end{Figure}

Finalmente, los datos cargados se pueden visualizar en el panel de gráficas a la derecha de la aplicación, como se observa en la figura \ref{fig:report_4}.

\begin{Figure}
    \centering
    \includegraphics[width=1\linewidth]{report-4.png}
     \captionof{figure}{Visualización de los resultados cargados.}
    \label{fig:report_4}
\end{Figure}

\subsection{Menú de ajuste de parámetros}

Similar al proceso de lectura de datos para experimentos realizados en el menú principal, para está ocasión se debe estar en el menú de ajuste de parámetros, como se muestra en la figura \ref{fig:tuning} y repetir el proceso explicado previamente pero cargando los resultados de algún experimento de ajuste de parámetros. En este caso se cargó uno llamado \emph{resultados\_2.npz}, que se visualiza en la figura \ref{fig:report_5}.

\begin{Figure}
    \centering
    \includegraphics[width=1\linewidth]{report-5.png}
     \captionof{figure}{Visualización de los resultados de ajuste de parámetros cargados.}
    \label{fig:report_5}
\end{Figure}

\section{Acerca de}

\begin{multicols}{2}

Como se mencionó al principio de este manual, esta aplicación se encuentra asociada con varias compañias colombianas relacionadas en el sector de la sísmica. Por lo tanto, si se pulsa el botón \emph{Acerca de} \ref{fig:about_tab}, se abrirá una nueva ventana donde se menciona dicha información de manera más detallada, como se observa en la figura \ref{fig:about_of}.

\begin{Figure}
    \centering
    \includegraphics[width=0.3\linewidth]{about-tab.png}
     \captionof{figure}{Botón de Acerca de.}
    \label{fig:about_tab}
\end{Figure}

\end{multicols}

\begin{Figure}
    \centering
    \includegraphics[width=1\linewidth]{about.png}
     \captionof{figure}{Acerca de.}
    \label{fig:about_of}
\end{Figure}

\section{Recomendaciones de uso}

Con el propósito de obtener el mayor rendimiento de la aplicación se realizan las siguientes recomendaciones:

\begin{itemize}
    \item Realizar los experimentos siguiendo este orden:
    
    \textbf{Menú principal}
    
    \begin{enumerate}
        \item Cargar un dato sísmico.
        \item Seleccionar el algoritmo que desea usar y configurar sus parámetros.
        \item Seleccionar el tipo de submuestreo y configurar sus parámetros.
        \item Darle un nombre a los resultados a guardar usando el botón \hspace{0.5mm} \faSave.
        \item Inicial el experimento usando el botón \hspace{0.5mm} \faPlay.
    \end{enumerate}
    
    \textbf{Ajuste de parámetros}
    
    Se sigue el mismo orden que el menú principal, pero los incisos 2 y 3 cambian por:
    
    \begin{enumerate}
        \item[2.] Seleccionar el agloritmo que desar usar y configurar la cantidad máxima de iteraciones.
        \item[3.] Configurar el ajuste de parámetros que se desea realizar.
    \end{enumerate}
    
    \item Para la configuración de parámetros se recomienda usar valores razonables y no excesivamente grandes debido a que dependiendo del equipo de cómputo utilizado y al algoritmo seleccionado, se puede elevar el consumo de CPU y memoria RAM al intentar utilizar otras herramientas de la aplicación mientras se corre algún experimento. Actualmente, la aplicación continua en desarrollo para soportar estos tipos de usos o por lo menos advirtiendo al usuario.
    
\end{itemize}


% \begin{thebibliography}{99}
% \bibitem{web1} \url{http://bit.ly/PhysLab_Link01}
% \bibitem{web2} \url{http://bit.ly/PhysLab_Link02}
% \end{thebibliography}
\end{document}